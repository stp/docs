% vim: shiftwidth=2:tabstop=2:softtabstop=2:expandtab:textwidth=120:colorcolumn=120
% \documentclass{llncs}
\documentclass{easychair}
%\documentclass[llncs]{IEEEtran}

\usepackage{microtype} %This gives MUCH better PDF results!
%\usepackage[active]{srcltx} %DVI search
% \usepackage[cmex10]{amsmath}
% \usepackage{amssymb}
\usepackage{fnbreak} %warn for split footnotes
\usepackage{url}
%\usepackage{qtree} %for drawing trees
%\usepackage{fancybox} % if we need rounded corners
%\usepackage{pict2e} % large circles can be drawn
%\usepackage{courier} %for using courier in texttt{}
%\usepackage{nth} %allows to \nth{4} to make 1st 2nd, etc.
%\usepackage{subfigure} %allows to have side-by-side figures
%\usepackage{booktabs} %nice tables
%\usepackage{multirow} %allow multiple cells with rows in tabular
\usepackage[utf8]{inputenc} % allows to write Faugere correctly
% \usepackage[bookmarks=true, citecolor=black, linkcolor=black, colorlinks=true]{hyperref}
% \hypersetup{
% pdfauthor ={STP},
% pdftitle = {STP},
% pdfsubject = {SMT Competition 2020},
% pdfkeywords = {SMT Solver},
% pdfcreator = {PdfLaTeX with hyperref package},
% pdfproducer = {PdfLaTex}}
%\usepackage{butterma}

%\usepackage{pstricks}
%\usepackage{graphicx,epsfig,xcolor}
%\usepackage[algoruled, linesnumbered, lined]{algorithm2e} %algorithms
\setlength{\parskip}{1ex}
\begin{document}
\title{STP in the SMTCOMP 2021}
\author{Various}
\institute{}

\maketitle
\thispagestyle{empty}
\pagestyle{empty}

\section{Background}
STP\cite{Vijay:Thesis:2007} is an open-source solver for QF\_BV and arrays without extensionality. 
STP recursively simplifies bit-vector constraints, solves linear bit-vector equations, and then eagerly encodes them to CNF for solving. 
Array axioms are added as needed during an abstraction-refinement phase.

STP was originally developed by Vijay Ganesh under the supervision of Professor David Dill. 
Later releases were developed by Trevor Hansen under the supervision of Peter Schachte and Harald Søndergaard. 
STP handles arbitrary precision integers using Steffen Beyer's library. 
STP encodes into CNF via the and-inverter graph package ABC of Alan Mishchenko~\cite{Brayton:2010:AAI:2144310.2144317}.
By default STP uses CryptoMiniSat~\cite{CMS:github}, but also uses MiniSat~\cite{MiniSat:github} and Riss~\cite{Riss:github}.

\section{Recent Developments to STP}
In the last year:

Andrew V. Jones and Mate Soos have considerably improved the build system. 
Mate Soos and Norbert Manthey have got the distributed and parallel versions of STP working. 
Trevor Hansen has sped up some simplifications. 

\section*{Acknowledgements}
Vijay Ganesh, Dan Liew, Mate Soos and Ryan Govostes contributed substantialy to the STP code base.

\bibliographystyle{splncs}
\bibliography{sigproc}

\vfill
\pagebreak

\end{document}

